% 4022 Project Thesis Template
% University of Cape Town Department of Electrical Engineering
% Template Version 0.1 - August 2021
% Adapted from the following 2 sources by J Wyngaard
% https://code.engineering.queensu.ca/ece/thesis-template-for-latex
% https://github.com/jpt13653903/LaTeX_UCT_Report.git

% -> FILE DESCRIPTION :: Preamble.tex
%	This files defines the imports necessary for proper compiling of the thesis document and create the styling of the document.

%*************************************************************************************************************
% DOCUMENT
%*************************************************************************************************************
% -> This section defines the basic thesis document. 

% Set Document Type
\documentclass[11pt]{report}  

% Select Thesis Package
%\usepackage{0_Config/1_Styles/UCTEEThesisStyle}
\usepackage{UCT_EE_ThesisStyle}

    
%*************************************************************************************************************
% PATCHING
%*************************************************************************************************************
% -> This section imports functionality which fixes Latex bugs

\usepackage{morewrites}
%\usepackage{scrwfile}

%*************************************************************************************************************
% SPACING
%*************************************************************************************************************
% -> This section defines the basic spacing options for the thesis document. 

%\usepackage{indentfirst} % Indent first paragraph after header
\setlength{\parindent}{0pt} % Set paragraph indentation to 0
% Skip a line in between paragraphs
\usepackage{parskip}
\usepackage{setspace}
\setstretch{1.20}

%*******************************************************************************
% HEADINGS
%*******************************************************************************
% -> This changes the headings go that they are prettier, this can be commented out for traditional headings.

\usepackage{sectsty}

%*************************************************************************************************************
% FONTS
%*************************************************************************************************************
% -> This section defines the fonts for various document entities.

\allsectionsfont{\bfseries}% set all the section font to bfseries
\chapterfont{\Huge} % set the sizes of chapters, sections ...
\sectionfont{\Large}
\subsectionfont{\large}

%*************************************************************************************************************
% FORMATTING
%*************************************************************************************************************
% -> This section defines formatting Table of Contents entr
% example: Chapter 1 Introduction

\usepackage[subfigure]{tocloft}
\usepackage{tocloft}
\usepackage{multicol}

\renewcommand{\cftchappresnum}{Chapter }
\renewcommand{\cftchapaftersnum}{:}
\renewcommand{\cftchapnumwidth}{7em}

% for formatting Table of Contents entry for Appendix, example: Appendix 1: Stuff
\newcommand*\updatechaptername{%
   \addtocontents{toc}{\protect\renewcommand*\protect\cftchappresnum{Appendix }}
}

%*******************************************************************************
% FOOTNOTES
%*******************************************************************************
% -> This section defines formating for footnotes

\interfootnotelinepenalty=10000 % This line stops footnotes from splitting onto two pages.

%*******************************************************************************
% VERBATIM
%*******************************************************************************
% -> 

\usepackage{moreverb}        % Using this package to get better control of the
                             % verbatim environment, mostly for the use of the
                             % listing environment which puts line number
                             % beside each line.  Note that there has to be a number
                             % in each set of brackets, i.e., \begin{listing}[1]{1}.
                             % PDF info file is "The moreverb package" by
                             % Robin Fairbairns (rf@cl.cam.ac.uk) after
                             % Angus Duggan, Rainer Schopf and Victor Eijkhout, 2000/06/29.
%-------------------------------------------------------------------------------
\usepackage{verbatim}        % allows the use of \begin{comment} and \end{comment}
                             % as well as \verbatiminput{file}
                             % Note:  when using verbatim to input from a text file,
                             % such as a specification or code, use \begin{singlespacing}
                             % and \end{singlespacing}.  Also, tabs are not read
                             % properly, so the input file must only use spaces.

%                             \begin{comment}
%                             Can also use the verbatim package for
%                             comments like this...
%                             \end{comment}


%*******************************************************************************
% GLOSSARY
%*******************************************************************************
% -> Creates the glossary

\usepackage[acronym,automake,toc]{glossaries}
\newglossary[slg]{symbol}{sym}{sdn}{List of Symbols}
\makeglossaries

% See Glossary/Glossary.tex for the content of your glossary

%*******************************************************************************
% INDEX
%*******************************************************************************
% -> Creates the Index

\usepackage{makeidx}         
\makeindex

%*******************************************************************************
% MATH
%*******************************************************************************
% -> Import and configure packages for math
\usepackage{mathrsfs}		 % For Symbols
\usepackage{amssymb}		 % For More Symbols
\usepackage{amsfonts}		 % For Other Symbols
\usepackage{amsmath}         % For Equations
\usepackage{amsthm}          % For Theorems & Definitions


% Using the amsthm package, define a new theorem environment for my
% definition.  * means don't number it.
\newtheorem*{definition}{Definition}
\usepackage{cases}           % to make numbered cases (equations)
\usepackage{calc}            % Used with the Ventry environment defined below.

%*******************************************************************************
% FLOATS, PLOTS, AND FIGURES
%*******************************************************************************
% -> This section defines and configures packages for various styles of floating objects and figures. 
 
\usepackage{graphicx}       % for graphic images (use \includegraphics[...]{file.eps})
%\usepackage{subfigure}      % for subfigures (figures within figures)
\usepackage{subfig}			% for subfigures (figures within figures). Same as subfigures except it works.
\usepackage{wrapfig}		% for wrapping text around figures
%\usepackage{boxedminipage}  % to make boxed minipages, i.e., boxes around figures
\usepackage{rotate}         % for use of \begin{sideways} and \end{sideways}
\usepackage{listings}       % for use of printing code blocks
\usepackage{pgfplots}		% for plots
\usepackage{pdfpages}		% for importing PDFs
%\usepackage{geometry}

% Define how the ``listings'' to look.
%\lstset{basicstyle=\footnotesize,, numbers=left, numberstyle=\small, stepnumber=1, numbersep=5pt, showspaces=false, lineskip=-1pt}

\usepackage{float}           % Using this package to get better control of floats
                             % including the ability to define new float types for
                             % specification and code listings.
                             % DVI info file is "An Improved Environment for Floats"
                             % by Anselm Lingnau, lingnau@tm.informatik.uni=frankfurt.de
                             % 1995/03/29.

% Style of float used for code listings
\floatstyle{ruled}
\newfloat{Listing}{H}{lis}[chapter]
\renewcommand{\lstlistlistingname}{List of Code Listings} % Title be changed as necessary

                             % Note:  The listings don't have space between the chapters, unlike
                             % the standard list of tables etc.  At the end, copy the spacing
                             % commands from the .toc file and insert into the .lis file.  Then,
                             % DO NOT LATEX it again, simply go to the DVI viewer!

\usepackage{placeins} % for \FloatBarrier which stops floats from crossing a point

%*******************************************************************************
% TABLES
%*******************************************************************************
% -> This section defines tables

\usepackage{tabularx}        % Package used to make variable width-columns, i.e.,
                             % column widths are changed to fit the maximum width
                             % and text is wrapped nicely.

\usepackage{threeparttable}

\usepackage{lscape}			% For landscape
\usepackage{adjustbox}		% To adjust table width
\usepackage{multirow}		% For merging and splitting table cells

%*******************************************************************************
% CAPTIONS
%*******************************************************************************
% ->  This section defines captions

\usepackage{caption}   % Package used to make my captions 'hang', i.e., wrap around, but not under the name of the caption.
%\usepackage[justification=centering]{caption} % Captions that are centered
                             

% Find that the captions are too far from my verbatim figures, but if
% I change it to 0, then the captions are too close for my other types
% of figures.  Maybe set each one separately?
%\setlength{\abovecaptionskip}{1ex}

%\setlength{\textfloatsep}{1ex plus1pt minus1pt}

%\setlength{\intextsep}{1ex plus1pt minus1pt}

%\setlength{\floatsep}{1ex plus1pt minus1pt}

%*******************************************************************************
% MISCELLANEOUS
%*******************************************************************************
% ->  This section defines miscellaneous packages that facilitate miscelaneous things

\usepackage{layout}          % useful for determining the margins of a document
                             % use with \layout command
                             
\usepackage{changebar}       % Way of indicating modifications by putting bars in the
                             % margin.  Read about it in "The Latex Companion".
                             
\usepackage{enumitem}		% For enumerated lists

\usepackage{siunitx}        % Used for typesetting units

%*******************************************************************************
% REFERENCES ETC.
%*******************************************************************************
% ->  This section defines the reference section

\usepackage{varioref}        % Better page references, e.g., "on preceding page", etc.
                             % \vref{key} Create an enhanced reference.
                             % \vpageref[text]{key} Create an enhanced page reference.
                             % \vrefrange{key}{key} Create an enhanced range of references.
                             % \vpagerefrange[text]{key}{key} Create an enhanced range of page references.
                             % Note: doesn't really work for consecutive pages.

% Renewing the text for before and after
\renewcommand{\reftextafter}{on the next page}
\renewcommand{\reftextbefore}{on the previous page}
\usepackage{url}             % for use of \url - pretty web addresses
\usepackage{fancyhdr}
\usepackage{cite}
\usepackage{notoccite}      % Prevents references from starting in TOC, List of Figures, etc.

%*******************************************************************************
% Block Diagrams
%*******************************************************************************
% ->  This section defines imports for block diagrams and flow charts

\usepackage{tikz}
\usetikzlibrary{arrows,automata,arrows.meta}

% Block Diagram
\tikzstyle{block} = [draw,fill=blue!20,minimum size=2em]
\def\radius{.7mm} 
\tikzstyle{branch}=[fill,shape=circle,minimum size=3pt,inner sep=0pt]

% Flow Charts
\usepackage{pgf}
\usepackage[utf8x]{inputenc}
\usetikzlibrary{positioning,calc}
\tikzset{
	state/.style={
		rectangle,
		rounded corners,
		draw=black, very thick,
		minimum height=2em,
		inner sep=2pt,
		text centered,
	},
}

%*******************************************************************************
% HYPERLINKS (must be last)
%*******************************************************************************
% ->  This section defines how hyperlinks function

% Uncomment these next two lines for linkback to citation pages in biblio
%%#\renewcommand*\backref[1]{\ifx#1\relax \else \linebreak Cited on page(s): #1. \fi}

\hypersetup{
	colorlinks=true,  % Change links to being coloured text, no boxes
	linkcolor=blue,
	urlcolor=blue,
    citecolor=blue
}
% Neat package to turn href, ref, cite, gloss entries
% into hyperlinks in the dvi file.
% Make sure this is the last package loaded.
% Use with dvips option to get hyperlinks to work in ps and pdf
% files.  Unfortunately, then they don't work in the dvi file!
% Use without the dvips option to get the links to work in the dvi file.

% Note:  \floatstyle{ruled} don't work properly; so change to plain.
% Not as pretty, but functional...
% The bookmarks option sets up proper bookmarks in the pdf file :)

% Need this command to allow hyperref to play nicely with gloss; otherwise
% almost every \gloss will cause an error...
%%#\renewcommand{\glosslinkborder}{0 0 0}

\usepackage[colorinlistoftodos]{todonotes}
\newcommand{\todoall}[1][]{\todo[color=yellow, inline]}

\usepackage{longtable}

\usepackage[table,xcdraw]{xcolor}




% Page geometry
\usepackage[a4paper,margin=20mm,top=25mm,bottom=25mm]{geometry}