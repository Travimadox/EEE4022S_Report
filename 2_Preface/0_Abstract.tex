\phantomsection
\prefacesection{Abstract}

% INSERT YOUR ABSTRACT %
In this work we describe a novel Thesis Template, to be used by students in Electrical \& Engineering at the University of Cape Town. This section entails the abstract of the document. 

%--------------------------------------------------------------------
% Taken from "Scientific Writing by D. Branch Moody"
%--------------------------------------------------------------------
% The abstract is 2 to 6 sentences long and is always less than 500 
% words.  Abstracts are increasingly less than 200 words, 
% according to journal rules. The abstract outlines three points: 
% the field's gap in knowledge, the outcomes of experiments and their
% impact on the field. 
%
% In learning to write abstracts for national meetings, you were 
% probably taught to include descriptions of individual experiments 
% and methods.  The "national meeting" abstract is your only 
% chance to communicate results to the reviewers, so is longer 
% and more detailed.
%
% The abstract submitted to a journal for publication is 
% immediately followed by all of the experiments, so the abstract 
% itself does not need to list every finding and method.  Thus, an  
% abstract for a paper submitted to a journal is usually more 
% conceptual and clearly states the take home point. Increasingly,
% journals have shortened the allowed length of abstracts with 
% many now in the range of 100 to 200 words. Journals now strongly
% encourage authors to focus on the conceptual advance rather 
% than the serial of listing individual experiments. Readers are
% usually interested in why the work is new and important and 
% not a detailed listing of the mouse strains and monoclonal
% antibodies used. 
%
% 'Current knowledge ends' statement. Experienced writers are 
% careful to say where knowledge ends in a statement that comes 
% before the explanation of the new data starts.  The "current 
% knowledge ends" statement, if well crafted, will make it self 
% evident how the new data in the submitted is original or important.
% For example, the second sentence of an abstract could say 
% "Current knowledge of human CD1-reactive T cells derives mainly 
% from study of individual T cell clones, but the roles of such 
% T cell in vivo are not known."  This creates the general 
% expectation that the data that follow are opening up new avenues, 
% yet still allows the reader to decide if the point is proven. 
%--------------------------------------------------------------------
Coming soon